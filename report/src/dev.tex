\section{Faces recognition process}

\paragraph{}{
    To recognize each faces, I use a perceptron trained to recognize a 
 specific face. So I have four neurons: One to recognize happy faces, one to
 recognize sad faces and so on.
}

\subsection{Perceptron}

\label{par:percep}
\paragraph{}{
    A perceptron, or a neuron, is basically a function encapsulate in a 
 cell. The cell has inputs, the synapses. Each synapse has a weight. To
 activate the neuron, the neuron uses his activation function. The activation
 function is the sum of all inputs, regarding the weights of each synapse.
 Then to adjust the activation level, a bias is added, it is a special
 input always at $1$. The bias has also a weight. Then we use the 
 \texttt{sigmoid} \footnote{ the \texttt{sigmoid} function is define by 
 $sigmoid(t) = \frac{1}{1 + e^{-t}}$} function to activate the neuron with the 
 previous sum as the input. The output level is mark out by $0$ and $1$. 
 Finally,the neuron is activated when his output is \textit{close} to $1$. This
 model was firstly proposed by McCulloch-Pitts\cite{art:mcp}.
}

\begin{figure}[!h]
    \begin{center}
        % Graphic for TeX using PGF
% Title: /home/benjamin/Documents/Université/M1-2015-2016 _ Umeå/Term 1/Fundamentals of Artificial Intelligence Fall/lab2/figures/perceptron.dia
% Creator: Dia v0.97.3
% CreationDate: Fri Oct 23 11:10:26 2015
% For: benjamin
% \usepackage{tikz}
% The following commands are not supported in PSTricks at present
% We define them conditionally, so when they are implemented,
% this pgf file will use them.
\ifx\du\undefined
  \newlength{\du}
\fi
\setlength{\du}{10\unitlength}
\begin{tikzpicture}
\pgftransformxscale{1.000000}
\pgftransformyscale{-1.000000}
\definecolor{dialinecolor}{rgb}{0.000000, 0.000000, 0.000000}
\pgfsetstrokecolor{dialinecolor}
\definecolor{dialinecolor}{rgb}{1.000000, 1.000000, 1.000000}
\pgfsetfillcolor{dialinecolor}
\pgfsetlinewidth{0.150000\du}
\pgfsetdash{}{0pt}
\pgfsetdash{}{0pt}
\pgfsetmiterjoin
\definecolor{dialinecolor}{rgb}{0.000000, 0.000000, 0.000000}
\pgfsetstrokecolor{dialinecolor}
\pgfpathellipse{\pgfpoint{55.000000\du}{-19.000000\du}}{\pgfpoint{9.000000\du}{0\du}}{\pgfpoint{0\du}{5.000000\du}}
\pgfusepath{stroke}
% setfont left to latex
\definecolor{dialinecolor}{rgb}{0.000000, 0.000000, 0.000000}
\pgfsetstrokecolor{dialinecolor}
\node at (55.000000\du,-18.847500\du){$ g( sigmoid(\sum_{i=1}^{400}(s_i \times w_i) + bias )) $};
\pgfsetlinewidth{0.150000\du}
\pgfsetdash{}{0pt}
\pgfsetdash{}{0pt}
\pgfsetbuttcap
{
\definecolor{dialinecolor}{rgb}{0.000000, 0.000000, 0.000000}
\pgfsetfillcolor{dialinecolor}
% was here!!!
\pgfsetarrowsend{to}
\definecolor{dialinecolor}{rgb}{0.000000, 0.000000, 0.000000}
\pgfsetstrokecolor{dialinecolor}
\draw (41.000000\du,-23.000000\du)--(46.685084\du,-20.913417\du);
}
\pgfsetlinewidth{0.150000\du}
\pgfsetdash{}{0pt}
\pgfsetdash{}{0pt}
\pgfsetbuttcap
{
\definecolor{dialinecolor}{rgb}{0.000000, 0.000000, 0.000000}
\pgfsetfillcolor{dialinecolor}
% was here!!!
\pgfsetarrowsend{to}
\definecolor{dialinecolor}{rgb}{0.000000, 0.000000, 0.000000}
\pgfsetstrokecolor{dialinecolor}
\draw (41.000000\du,-25.000000\du)--(48.636039\du,-22.535534\du);
}
\pgfsetlinewidth{0.150000\du}
\pgfsetdash{}{0pt}
\pgfsetdash{}{0pt}
\pgfsetbuttcap
{
\definecolor{dialinecolor}{rgb}{0.000000, 0.000000, 0.000000}
\pgfsetfillcolor{dialinecolor}
% was here!!!
\pgfsetarrowsend{to}
\definecolor{dialinecolor}{rgb}{0.000000, 0.000000, 0.000000}
\pgfsetstrokecolor{dialinecolor}
\draw (41.000000\du,-15.000000\du)--(46.685084\du,-17.086583\du);
}
\pgfsetlinewidth{0.150000\du}
\pgfsetdash{}{0pt}
\pgfsetdash{}{0pt}
\pgfsetbuttcap
{
\definecolor{dialinecolor}{rgb}{0.000000, 0.000000, 0.000000}
\pgfsetfillcolor{dialinecolor}
% was here!!!
\pgfsetarrowsend{to}
\definecolor{dialinecolor}{rgb}{0.000000, 0.000000, 0.000000}
\pgfsetstrokecolor{dialinecolor}
\draw (41.000000\du,-13.000000\du)--(48.636039\du,-15.464466\du);
}
\pgfsetlinewidth{0.150000\du}
\pgfsetdash{{1.000000\du}{1.000000\du}}{0\du}
\pgfsetdash{{0.600000\du}{0.600000\du}}{0\du}
\pgfsetbuttcap
{
\definecolor{dialinecolor}{rgb}{0.000000, 0.000000, 0.000000}
\pgfsetfillcolor{dialinecolor}
% was here!!!
\definecolor{dialinecolor}{rgb}{0.000000, 0.000000, 0.000000}
\pgfsetstrokecolor{dialinecolor}
\draw (41.000000\du,-21.000000\du)--(41.000000\du,-17.000000\du);
}
\pgfsetlinewidth{0.150000\du}
\pgfsetdash{}{0pt}
\pgfsetdash{}{0pt}
\pgfsetbuttcap
{
\definecolor{dialinecolor}{rgb}{0.000000, 0.000000, 0.000000}
\pgfsetfillcolor{dialinecolor}
% was here!!!
\pgfsetarrowsend{to}
\definecolor{dialinecolor}{rgb}{0.000000, 0.000000, 0.000000}
\pgfsetstrokecolor{dialinecolor}
\draw (64.000000\du,-19.000000\du)--(68.000000\du,-19.000000\du);
}
\pgfsetmiterjoin
\pgfsetdash{}{0pt}
\pgfsetlinewidth{0.200000\du}
\definecolor{dialinecolor}{rgb}{0.000000, 0.000000, 0.000000}
\pgfsetstrokecolor{dialinecolor}
\draw (39.000000\du,-23.666667\du)--(41.000000\du,-23.666667\du);
\definecolor{dialinecolor}{rgb}{0.000000, 0.000000, 0.000000}
\pgfsetstrokecolor{dialinecolor}
\draw (39.000000\du,-22.333333\du)--(41.000000\du,-22.333333\du);
\pgfsetlinewidth{0.200000\du}
\definecolor{dialinecolor}{rgb}{0.000000, 0.000000, 0.000000}
\pgfsetstrokecolor{dialinecolor}
\draw (39.000000\du,-25.000000\du)--(39.000000\du,-21.000000\du)--(41.000000\du,-21.000000\du)--(41.000000\du,-25.000000\du)--cycle;
\pgfsetlinewidth{0.150000\du}
\pgfsetdash{{1.000000\du}{1.000000\du}}{0\du}
\pgfsetdash{{0.600000\du}{0.600000\du}}{0\du}
\pgfsetbuttcap
{
\definecolor{dialinecolor}{rgb}{0.000000, 0.000000, 0.000000}
\pgfsetfillcolor{dialinecolor}
% was here!!!
\definecolor{dialinecolor}{rgb}{0.000000, 0.000000, 0.000000}
\pgfsetstrokecolor{dialinecolor}
\draw (39.000000\du,-21.000000\du)--(39.000000\du,-17.000000\du);
}
% setfont left to latex
\definecolor{dialinecolor}{rgb}{0.000000, 0.000000, 0.000000}
\pgfsetstrokecolor{dialinecolor}
\node at (40.000000\du,-12.000000\du){Image};
% setfont left to latex
\definecolor{dialinecolor}{rgb}{0.000000, 0.000000, 0.000000}
\pgfsetstrokecolor{dialinecolor}
\node at (55.000000\du,-12.000000\du){Perceptron};
% setfont left to latex
\definecolor{dialinecolor}{rgb}{0.000000, 0.000000, 0.000000}
\pgfsetstrokecolor{dialinecolor}
\node at (66.000000\du,-21.000000\du){output};
% setfont left to latex
\definecolor{dialinecolor}{rgb}{0.000000, 0.000000, 0.000000}
\pgfsetstrokecolor{dialinecolor}
\node at (45.000000\du,-25.000000\du){inputs};
\pgfsetmiterjoin
\pgfsetdash{}{0pt}
\pgfsetlinewidth{0.200000\du}
\definecolor{dialinecolor}{rgb}{0.000000, 0.000000, 0.000000}
\pgfsetstrokecolor{dialinecolor}
\draw (39.000000\du,-15.666667\du)--(41.000000\du,-15.666667\du);
\definecolor{dialinecolor}{rgb}{0.000000, 0.000000, 0.000000}
\pgfsetstrokecolor{dialinecolor}
\draw (39.000000\du,-14.333333\du)--(41.000000\du,-14.333333\du);
\pgfsetlinewidth{0.200000\du}
\definecolor{dialinecolor}{rgb}{0.000000, 0.000000, 0.000000}
\pgfsetstrokecolor{dialinecolor}
\draw (39.000000\du,-17.000000\du)--(39.000000\du,-13.000000\du)--(41.000000\du,-13.000000\du)--(41.000000\du,-17.000000\du)--cycle;
\end{tikzpicture}

    \end{center}
    \caption{\label{fig:neuron} An artificial neuron}
\end{figure}

\paragraph{}{
    So, a neuron can be seen as this mathematical expression :
 \begin{equation}
    g( sigmoid(\sum_{i=1}^{400}(s_i \times w_i) + bias ))
 \end{equation}
 The figure \ref{fig:neuron} shows how we can modeled an artificial neuron.
 This model is implemented in a class, \texttt{neuron}. Such source code is 
 available appendix \ref{app:neuron}. A neuron is created of a given amount of 
 inputs and has as many synapses as inputs size. The most import functions are 
 \texttt{g} and \texttt{learn}.
}
    \subparagraph{\texttt{learn}}{
     This function allows us to train a neuron. To do so, the neuron updates the
 weight of each synapse and of the bias regarding a given learning rate 
 $\alpha$, inputs and desire output for the inputs. Which can translate to the
 algorithm below at the figure \ref{fig:algo_learn}.
}

\begin{figure}[!h]
    \begin{algorithmic}
    \Function{learn}{inputs, output}
        \State $error \gets output - g(inputs)$
        \For{ i from 0 to synapses\_number}
            \State $ synapse_i \gets synapse_i + input_i \times error \times \alpha $
        \EndFor
        \State $bias \gets bias \times error \times \alpha $
     \EndFunction
    \end{algorithmic}
    \caption{\label{fig:algo_learn} Learning algorithm for neurons.}
\end{figure}

    \subparagraph{\texttt{g}}{
     The activation function, \texttt{g} define in which state is the neuron
 for given inputs. His operating is explained above in part \ref{par:percep}.
 The figure \ref{fig:algo_g} shows the algorithm used to implement the activation
 function.
}

\begin{figure}[!h]
    \begin{algorithmic}
    \Function{g}{inputs}
        \State $sum \gets 0$
        \For{ i from 0 to synapses\_number}
            \State $ sum \gets sum + input_i \times synapse_i $
        \EndFor
        \State $ sum \gets sum + bias $
        \State \Return \Call{sigmoid}{sum}
     \EndFunction
    \end{algorithmic}
    \caption{\label{fig:algo_g} Activation algorithm for neurons.}
\end{figure}


\paragraph{}{
    But using only one neuron is not interesting. As we want to recognize four
 different face types, I use a neuron for each face. So I build a network of 
 four neurons and I train each neurons to recognize only one face type.
}

\subsection{The network}

\paragraph{Recognize faces}{
    The network I realized to recognize faces can be seen figure \ref{fig:ann}.
 Each perceptrons have as many inputs as there are pixels on images. To know
 which faces is recognized, I get the neuron number with the highest activation
 level. This number matches with the faces the network should recognize.
}

\begin{figure}[!h]
    \begin{center}
        % Graphic for TeX using PGF
% Title: /home/benjamin/Documents/Université/M1-2015-2016 _ Umeå/Term 1/Fundamentals of Artificial Intelligence Fall/lab2/figures/ann.dia
% Creator: Dia v0.97.3
% CreationDate: Fri Oct 23 11:34:38 2015
% For: benjamin
% \usepackage{tikz}
% The following commands are not supported in PSTricks at present
% We define them conditionally, so when they are implemented,
% this pgf file will use them.
\ifx\du\undefined
  \newlength{\du}
\fi
\setlength{\du}{13\unitlength}
\begin{tikzpicture}[scale=0.8]
\pgftransformxscale{1.000000}
\pgftransformyscale{-1.000000}
\definecolor{dialinecolor}{rgb}{0.000000, 0.000000, 0.000000}
\pgfsetstrokecolor{dialinecolor}
\definecolor{dialinecolor}{rgb}{1.000000, 1.000000, 1.000000}
\pgfsetfillcolor{dialinecolor}
\pgfsetlinewidth{0.050000\du}
\pgfsetdash{}{0pt}
\pgfsetdash{}{0pt}
\pgfsetbuttcap
{
\definecolor{dialinecolor}{rgb}{0.000000, 0.000000, 0.000000}
\pgfsetfillcolor{dialinecolor}
% was here!!!
\pgfsetarrowsend{to}
\definecolor{dialinecolor}{rgb}{0.000000, 0.000000, 0.000000}
\pgfsetstrokecolor{dialinecolor}
\draw (42.000000\du,-33.000000\du)--(49.000000\du,-32.000000\du);
}
\pgfsetlinewidth{0.050000\du}
\pgfsetdash{}{0pt}
\pgfsetdash{}{0pt}
\pgfsetbuttcap
{
\definecolor{dialinecolor}{rgb}{0.000000, 0.000000, 0.000000}
\pgfsetfillcolor{dialinecolor}
% was here!!!
\pgfsetarrowsend{to}
\definecolor{dialinecolor}{rgb}{0.000000, 0.000000, 0.000000}
\pgfsetstrokecolor{dialinecolor}
\draw (42.000000\du,-33.000000\du)--(49.000000\du,-22.000000\du);
}
\pgfsetlinewidth{0.100000\du}
\pgfsetdash{{1.000000\du}{1.000000\du}}{0\du}
\pgfsetdash{{0.600000\du}{0.600000\du}}{0\du}
\pgfsetbuttcap
{
\definecolor{dialinecolor}{rgb}{0.000000, 0.000000, 0.000000}
\pgfsetfillcolor{dialinecolor}
% was here!!!
\definecolor{dialinecolor}{rgb}{0.000000, 0.000000, 0.000000}
\pgfsetstrokecolor{dialinecolor}
\draw (42.000000\du,-30.000000\du)--(42.000000\du,-19.000000\du);
}
\pgfsetlinewidth{0.050000\du}
\pgfsetdash{}{0pt}
\pgfsetdash{}{0pt}
\pgfsetbuttcap
{
\definecolor{dialinecolor}{rgb}{0.000000, 0.000000, 0.000000}
\pgfsetfillcolor{dialinecolor}
% was here!!!
\pgfsetarrowsend{to}
\definecolor{dialinecolor}{rgb}{0.000000, 0.000000, 0.000000}
\pgfsetstrokecolor{dialinecolor}
\draw (55.000000\du,-32.000000\du)--(59.000000\du,-32.000000\du);
}
\pgfsetmiterjoin
\pgfsetdash{}{0pt}
\pgfsetlinewidth{0.100000\du}
\definecolor{dialinecolor}{rgb}{0.000000, 0.000000, 0.000000}
\pgfsetstrokecolor{dialinecolor}
\draw (40.000000\du,-32.666667\du)--(42.000000\du,-32.666667\du);
\definecolor{dialinecolor}{rgb}{0.000000, 0.000000, 0.000000}
\pgfsetstrokecolor{dialinecolor}
\draw (40.000000\du,-31.333333\du)--(42.000000\du,-31.333333\du);
\pgfsetlinewidth{0.100000\du}
\definecolor{dialinecolor}{rgb}{0.000000, 0.000000, 0.000000}
\pgfsetstrokecolor{dialinecolor}
\draw (40.000000\du,-34.000000\du)--(40.000000\du,-30.000000\du)--(42.000000\du,-30.000000\du)--(42.000000\du,-34.000000\du)--cycle;
\pgfsetlinewidth{0.100000\du}
\pgfsetdash{{1.000000\du}{1.000000\du}}{0\du}
\pgfsetdash{{0.600000\du}{0.600000\du}}{0\du}
\pgfsetbuttcap
{
\definecolor{dialinecolor}{rgb}{0.000000, 0.000000, 0.000000}
\pgfsetfillcolor{dialinecolor}
% was here!!!
\definecolor{dialinecolor}{rgb}{0.000000, 0.000000, 0.000000}
\pgfsetstrokecolor{dialinecolor}
\draw (40.000000\du,-30.000000\du)--(40.000000\du,-19.000000\du);
}
% setfont left to latex
\definecolor{dialinecolor}{rgb}{0.000000, 0.000000, 0.000000}
\pgfsetstrokecolor{dialinecolor}
\node at (41.000000\du,-13.777500\du){Image};
\pgfsetmiterjoin
\pgfsetdash{}{0pt}
\pgfsetlinewidth{0.100000\du}
\definecolor{dialinecolor}{rgb}{0.000000, 0.000000, 0.000000}
\pgfsetstrokecolor{dialinecolor}
\draw (40.000000\du,-17.666667\du)--(42.000000\du,-17.666667\du);
\definecolor{dialinecolor}{rgb}{0.000000, 0.000000, 0.000000}
\pgfsetstrokecolor{dialinecolor}
\draw (40.000000\du,-16.333333\du)--(42.000000\du,-16.333333\du);
\pgfsetlinewidth{0.100000\du}
\definecolor{dialinecolor}{rgb}{0.000000, 0.000000, 0.000000}
\pgfsetstrokecolor{dialinecolor}
\draw (40.000000\du,-19.000000\du)--(40.000000\du,-15.000000\du)--(42.000000\du,-15.000000\du)--(42.000000\du,-19.000000\du)--cycle;
\pgfsetlinewidth{0.050000\du}
\pgfsetdash{}{0pt}
\pgfsetdash{}{0pt}
\pgfsetbuttcap
{
\definecolor{dialinecolor}{rgb}{0.000000, 0.000000, 0.000000}
\pgfsetfillcolor{dialinecolor}
% was here!!!
\pgfsetarrowsend{to}
\definecolor{dialinecolor}{rgb}{0.000000, 0.000000, 0.000000}
\pgfsetstrokecolor{dialinecolor}
\draw (55.000000\du,-27.000000\du)--(59.000000\du,-27.000000\du);
}
\pgfsetlinewidth{0.120000\du}
\pgfsetdash{}{0pt}
\pgfsetdash{}{0pt}
\pgfsetmiterjoin
\definecolor{dialinecolor}{rgb}{0.000000, 0.000000, 0.000000}
\pgfsetstrokecolor{dialinecolor}
\pgfpathellipse{\pgfpoint{52.000000\du}{-27.000000\du}}{\pgfpoint{3.000000\du}{0\du}}{\pgfpoint{0\du}{2.000000\du}}
\pgfusepath{stroke}
% setfont left to latex
\definecolor{dialinecolor}{rgb}{0.000000, 0.000000, 0.000000}
\pgfsetstrokecolor{dialinecolor}
\node at (52.000000\du,-26.805000\du){sad};
\pgfsetlinewidth{0.120000\du}
\pgfsetdash{}{0pt}
\pgfsetdash{}{0pt}
\pgfsetmiterjoin
\definecolor{dialinecolor}{rgb}{0.000000, 0.000000, 0.000000}
\pgfsetstrokecolor{dialinecolor}
\pgfpathellipse{\pgfpoint{52.000000\du}{-32.000000\du}}{\pgfpoint{3.000000\du}{0\du}}{\pgfpoint{0\du}{2.000000\du}}
\pgfusepath{stroke}
% setfont left to latex
\definecolor{dialinecolor}{rgb}{0.000000, 0.000000, 0.000000}
\pgfsetstrokecolor{dialinecolor}
\node at (52.000000\du,-31.805000\du){happy};
\pgfsetlinewidth{0.120000\du}
\pgfsetdash{}{0pt}
\pgfsetdash{}{0pt}
\pgfsetmiterjoin
\definecolor{dialinecolor}{rgb}{0.000000, 0.000000, 0.000000}
\pgfsetstrokecolor{dialinecolor}
\pgfpathellipse{\pgfpoint{52.000000\du}{-22.000000\du}}{\pgfpoint{3.000000\du}{0\du}}{\pgfpoint{0\du}{2.000000\du}}
\pgfusepath{stroke}
% setfont left to latex
\definecolor{dialinecolor}{rgb}{0.000000, 0.000000, 0.000000}
\pgfsetstrokecolor{dialinecolor}
\node at (52.000000\du,-21.805000\du){mad};
\pgfsetlinewidth{0.120000\du}
\pgfsetdash{}{0pt}
\pgfsetdash{}{0pt}
\pgfsetmiterjoin
\definecolor{dialinecolor}{rgb}{0.000000, 0.000000, 0.000000}
\pgfsetstrokecolor{dialinecolor}
\pgfpathellipse{\pgfpoint{52.000000\du}{-17.000000\du}}{\pgfpoint{3.000000\du}{0\du}}{\pgfpoint{0\du}{2.000000\du}}
\pgfusepath{stroke}
% setfont left to latex
\definecolor{dialinecolor}{rgb}{0.000000, 0.000000, 0.000000}
\pgfsetstrokecolor{dialinecolor}
\node at (52.000000\du,-16.805000\du){mischievous};
\pgfsetlinewidth{0.050000\du}
\pgfsetdash{}{0pt}
\pgfsetdash{}{0pt}
\pgfsetbuttcap
{
\definecolor{dialinecolor}{rgb}{0.000000, 0.000000, 0.000000}
\pgfsetfillcolor{dialinecolor}
% was here!!!
\pgfsetarrowsend{to}
\definecolor{dialinecolor}{rgb}{0.000000, 0.000000, 0.000000}
\pgfsetstrokecolor{dialinecolor}
\draw (55.000000\du,-22.000000\du)--(59.000000\du,-22.000000\du);
}
\pgfsetlinewidth{0.050000\du}
\pgfsetdash{}{0pt}
\pgfsetdash{}{0pt}
\pgfsetbuttcap
{
\definecolor{dialinecolor}{rgb}{0.000000, 0.000000, 0.000000}
\pgfsetfillcolor{dialinecolor}
% was here!!!
\pgfsetarrowsend{to}
\definecolor{dialinecolor}{rgb}{0.000000, 0.000000, 0.000000}
\pgfsetstrokecolor{dialinecolor}
\draw (55.000000\du,-17.000000\du)--(59.000000\du,-17.000000\du);
}
\pgfsetlinewidth{0.050000\du}
\pgfsetdash{}{0pt}
\pgfsetdash{}{0pt}
\pgfsetbuttcap
{
\definecolor{dialinecolor}{rgb}{0.000000, 0.000000, 0.000000}
\pgfsetfillcolor{dialinecolor}
% was here!!!
\pgfsetarrowsend{to}
\definecolor{dialinecolor}{rgb}{0.000000, 0.000000, 0.000000}
\pgfsetstrokecolor{dialinecolor}
\draw (42.000000\du,-33.000000\du)--(49.000000\du,-27.000000\du);
}
\pgfsetlinewidth{0.050000\du}
\pgfsetdash{}{0pt}
\pgfsetdash{}{0pt}
\pgfsetbuttcap
{
\definecolor{dialinecolor}{rgb}{0.000000, 0.000000, 0.000000}
\pgfsetfillcolor{dialinecolor}
% was here!!!
\pgfsetarrowsend{to}
\definecolor{dialinecolor}{rgb}{0.000000, 0.000000, 0.000000}
\pgfsetstrokecolor{dialinecolor}
\draw (42.000000\du,-33.000000\du)--(49.000000\du,-17.000000\du);
}
\pgfsetlinewidth{0.050000\du}
\pgfsetdash{}{0pt}
\pgfsetdash{}{0pt}
\pgfsetbuttcap
{
\definecolor{dialinecolor}{rgb}{0.000000, 0.000000, 0.000000}
\pgfsetfillcolor{dialinecolor}
% was here!!!
\pgfsetarrowsend{to}
\definecolor{dialinecolor}{rgb}{0.000000, 0.000000, 0.000000}
\pgfsetstrokecolor{dialinecolor}
\draw (42.000000\du,-32.000000\du)--(49.000000\du,-27.000000\du);
}
\pgfsetlinewidth{0.050000\du}
\pgfsetdash{}{0pt}
\pgfsetdash{}{0pt}
\pgfsetbuttcap
{
\definecolor{dialinecolor}{rgb}{0.000000, 0.000000, 0.000000}
\pgfsetfillcolor{dialinecolor}
% was here!!!
\pgfsetarrowsend{to}
\definecolor{dialinecolor}{rgb}{0.000000, 0.000000, 0.000000}
\pgfsetstrokecolor{dialinecolor}
\draw (42.000000\du,-32.000000\du)--(49.000000\du,-22.000000\du);
}
\pgfsetlinewidth{0.050000\du}
\pgfsetdash{}{0pt}
\pgfsetdash{}{0pt}
\pgfsetbuttcap
{
\definecolor{dialinecolor}{rgb}{0.000000, 0.000000, 0.000000}
\pgfsetfillcolor{dialinecolor}
% was here!!!
\pgfsetarrowsend{to}
\definecolor{dialinecolor}{rgb}{0.000000, 0.000000, 0.000000}
\pgfsetstrokecolor{dialinecolor}
\draw (42.000000\du,-32.000000\du)--(49.000000\du,-32.000000\du);
}
\pgfsetlinewidth{0.050000\du}
\pgfsetdash{}{0pt}
\pgfsetdash{}{0pt}
\pgfsetbuttcap
{
\definecolor{dialinecolor}{rgb}{0.000000, 0.000000, 0.000000}
\pgfsetfillcolor{dialinecolor}
% was here!!!
\pgfsetarrowsend{to}
\definecolor{dialinecolor}{rgb}{0.000000, 0.000000, 0.000000}
\pgfsetstrokecolor{dialinecolor}
\draw (42.000000\du,-32.000000\du)--(49.000000\du,-17.000000\du);
}
\pgfsetlinewidth{0.050000\du}
\pgfsetdash{}{0pt}
\pgfsetdash{}{0pt}
\pgfsetbuttcap
{
\definecolor{dialinecolor}{rgb}{0.000000, 0.000000, 0.000000}
\pgfsetfillcolor{dialinecolor}
% was here!!!
\pgfsetarrowsend{to}
\definecolor{dialinecolor}{rgb}{0.000000, 0.000000, 0.000000}
\pgfsetstrokecolor{dialinecolor}
\draw (42.000000\du,-16.000000\du)--(49.000000\du,-22.000000\du);
}
\pgfsetlinewidth{0.050000\du}
\pgfsetdash{}{0pt}
\pgfsetdash{}{0pt}
\pgfsetbuttcap
{
\definecolor{dialinecolor}{rgb}{0.000000, 0.000000, 0.000000}
\pgfsetfillcolor{dialinecolor}
% was here!!!
\pgfsetarrowsend{to}
\definecolor{dialinecolor}{rgb}{0.000000, 0.000000, 0.000000}
\pgfsetstrokecolor{dialinecolor}
\draw (42.000000\du,-17.000000\du)--(49.000000\du,-17.000000\du);
}
\pgfsetlinewidth{0.050000\du}
\pgfsetdash{}{0pt}
\pgfsetdash{}{0pt}
\pgfsetbuttcap
{
\definecolor{dialinecolor}{rgb}{0.000000, 0.000000, 0.000000}
\pgfsetfillcolor{dialinecolor}
% was here!!!
\pgfsetarrowsend{to}
\definecolor{dialinecolor}{rgb}{0.000000, 0.000000, 0.000000}
\pgfsetstrokecolor{dialinecolor}
\draw (42.000000\du,-16.000000\du)--(49.000000\du,-27.000000\du);
}
\pgfsetlinewidth{0.050000\du}
\pgfsetdash{}{0pt}
\pgfsetdash{}{0pt}
\pgfsetbuttcap
{
\definecolor{dialinecolor}{rgb}{0.000000, 0.000000, 0.000000}
\pgfsetfillcolor{dialinecolor}
% was here!!!
\pgfsetarrowsend{to}
\definecolor{dialinecolor}{rgb}{0.000000, 0.000000, 0.000000}
\pgfsetstrokecolor{dialinecolor}
\draw (42.000000\du,-17.000000\du)--(49.000000\du,-27.000000\du);
}
\pgfsetlinewidth{0.050000\du}
\pgfsetdash{}{0pt}
\pgfsetdash{}{0pt}
\pgfsetbuttcap
{
\definecolor{dialinecolor}{rgb}{0.000000, 0.000000, 0.000000}
\pgfsetfillcolor{dialinecolor}
% was here!!!
\pgfsetarrowsend{to}
\definecolor{dialinecolor}{rgb}{0.000000, 0.000000, 0.000000}
\pgfsetstrokecolor{dialinecolor}
\draw (42.000000\du,-16.000000\du)--(49.000000\du,-32.000000\du);
}
\pgfsetlinewidth{0.050000\du}
\pgfsetdash{}{0pt}
\pgfsetdash{}{0pt}
\pgfsetbuttcap
{
\definecolor{dialinecolor}{rgb}{0.000000, 0.000000, 0.000000}
\pgfsetfillcolor{dialinecolor}
% was here!!!
\pgfsetarrowsend{to}
\definecolor{dialinecolor}{rgb}{0.000000, 0.000000, 0.000000}
\pgfsetstrokecolor{dialinecolor}
\draw (42.000000\du,-17.000000\du)--(49.000000\du,-22.000000\du);
}
\pgfsetlinewidth{0.050000\du}
\pgfsetdash{}{0pt}
\pgfsetdash{}{0pt}
\pgfsetbuttcap
{
\definecolor{dialinecolor}{rgb}{0.000000, 0.000000, 0.000000}
\pgfsetfillcolor{dialinecolor}
% was here!!!
\pgfsetarrowsend{to}
\definecolor{dialinecolor}{rgb}{0.000000, 0.000000, 0.000000}
\pgfsetstrokecolor{dialinecolor}
\draw (42.000000\du,-16.000000\du)--(49.000000\du,-17.000000\du);
}
\pgfsetlinewidth{0.050000\du}
\pgfsetdash{}{0pt}
\pgfsetdash{}{0pt}
\pgfsetbuttcap
{
\definecolor{dialinecolor}{rgb}{0.000000, 0.000000, 0.000000}
\pgfsetfillcolor{dialinecolor}
% was here!!!
\pgfsetarrowsend{to}
\definecolor{dialinecolor}{rgb}{0.000000, 0.000000, 0.000000}
\pgfsetstrokecolor{dialinecolor}
\draw (42.000000\du,-17.000000\du)--(49.000000\du,-32.000000\du);
}
\pgfsetlinewidth{0.050000\du}
\pgfsetdash{}{0pt}
\pgfsetdash{}{0pt}
\pgfsetmiterjoin
\definecolor{dialinecolor}{rgb}{0.000000, 0.000000, 0.000000}
\pgfsetstrokecolor{dialinecolor}
\draw (59.000000\du,-33.000000\du)--(59.000000\du,-16.000000\du)--(62.000000\du,-16.000000\du)--(62.000000\du,-33.000000\du)--cycle;
% setfont left to latex
\definecolor{dialinecolor}{rgb}{0.000000, 0.000000, 0.000000}
\pgfsetstrokecolor{dialinecolor}
\node at (60.500000\du,-24.305000\du){MAX};
\pgfsetlinewidth{0.050000\du}
\pgfsetdash{}{0pt}
\pgfsetdash{}{0pt}
\pgfsetbuttcap
{
\definecolor{dialinecolor}{rgb}{0.000000, 0.000000, 0.000000}
\pgfsetfillcolor{dialinecolor}
% was here!!!
\pgfsetarrowsend{to}
\definecolor{dialinecolor}{rgb}{0.000000, 0.000000, 0.000000}
\pgfsetstrokecolor{dialinecolor}
\draw (62.000000\du,-24.000000\du)--(66.000000\du,-24.000000\du);
}
% setfont left to latex
\definecolor{dialinecolor}{rgb}{0.000000, 0.000000, 0.000000}
\pgfsetstrokecolor{dialinecolor}
\node at (64.000000\du,-24.777500\du){face type};
\end{tikzpicture}

    \end{center}
    \caption{\label{fig:ann} The artificial neural network used to recognize faces}
\end{figure}

\paragraph{}{
    For instance, the first neuron has to be activated and with highest level 
 measured within the network if the inputs represented a happy face. To do so,
 I have to train the network before using it.
}

\paragraph{Training}{
    To train the network, I give a image to the network and if the wrong neuron
 is activated, I train each neuron to recognize the face. For the neuron which
 should be activated, the desired output is $1$ and for the others one, the 
 output should be $0$.
}

\paragraph{}{
    The training process is performed till the network recognize less than 
 eighty percent of the given images. As we have a input file with three hundred
 images with the corresponding faces, I train the network with two hundred images
 (approximately seventy five percent) and I measure the error rate with the 
 left images. Then I shuffle the two sets to redo a round training if the level 
 of recognized faces rate is too low.
}
